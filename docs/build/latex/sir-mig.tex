%% Generated by Sphinx.
\def\sphinxdocclass{report}
\documentclass[letterpaper,10pt,english]{sphinxmanual}
\ifdefined\pdfpxdimen
   \let\sphinxpxdimen\pdfpxdimen\else\newdimen\sphinxpxdimen
\fi \sphinxpxdimen=.75bp\relax
\ifdefined\pdfimageresolution
    \pdfimageresolution= \numexpr \dimexpr1in\relax/\sphinxpxdimen\relax
\fi
%% let collapsible pdf bookmarks panel have high depth per default
\PassOptionsToPackage{bookmarksdepth=5}{hyperref}

\PassOptionsToPackage{booktabs}{sphinx}
\PassOptionsToPackage{colorrows}{sphinx}

\PassOptionsToPackage{warn}{textcomp}
\usepackage[utf8]{inputenc}
\ifdefined\DeclareUnicodeCharacter
% support both utf8 and utf8x syntaxes
  \ifdefined\DeclareUnicodeCharacterAsOptional
    \def\sphinxDUC#1{\DeclareUnicodeCharacter{"#1}}
  \else
    \let\sphinxDUC\DeclareUnicodeCharacter
  \fi
  \sphinxDUC{00A0}{\nobreakspace}
  \sphinxDUC{2500}{\sphinxunichar{2500}}
  \sphinxDUC{2502}{\sphinxunichar{2502}}
  \sphinxDUC{2514}{\sphinxunichar{2514}}
  \sphinxDUC{251C}{\sphinxunichar{251C}}
  \sphinxDUC{2572}{\textbackslash}
\fi
\usepackage{cmap}
\usepackage[T1]{fontenc}
\usepackage{amsmath,amssymb,amstext}
\usepackage{babel}



\usepackage{tgtermes}
\usepackage{tgheros}
\renewcommand{\ttdefault}{txtt}



\usepackage[Bjarne]{fncychap}
\usepackage{sphinx}

\fvset{fontsize=auto}
\usepackage{geometry}


% Include hyperref last.
\usepackage{hyperref}
% Fix anchor placement for figures with captions.
\usepackage{hypcap}% it must be loaded after hyperref.
% Set up styles of URL: it should be placed after hyperref.
\urlstyle{same}

\addto\captionsenglish{\renewcommand{\contentsname}{Contents:}}

\usepackage{sphinxmessages}
\setcounter{tocdepth}{1}



\title{SIR\sphinxhyphen{}MIG}
\date{Apr 29, 2024}
\release{}
\author{Mike Moser}
\newcommand{\sphinxlogo}{\vbox{}}
\renewcommand{\releasename}{}
\makeindex
\begin{document}

\ifdefined\shorthandoff
  \ifnum\catcode`\=\string=\active\shorthandoff{=}\fi
  \ifnum\catcode`\"=\active\shorthandoff{"}\fi
\fi

\pagestyle{empty}
\sphinxmaketitle
\pagestyle{plain}
\sphinxtableofcontents
\pagestyle{normal}
\phantomsection\label{\detokenize{index::doc}}



\chapter{Indices and tables}
\label{\detokenize{index:indices-and-tables}}\begin{itemize}
\item {} 
\sphinxAtStartPar
\DUrole{xref,std,std-ref}{genindex}

\item {} 
\sphinxAtStartPar
\DUrole{xref,std,std-ref}{modindex}

\item {} 
\sphinxAtStartPar
\DUrole{xref,std,std-ref}{search}

\end{itemize}


\chapter{Welcome to SIR\sphinxhyphen{}MIG’s documentation!}
\label{\detokenize{index:welcome-to-sir-mig-s-documentation}}
\sphinxAtStartPar
\sphinxstylestrong{SIR\sphinxhyphen{}MIG} is an algorithm based on the inversion code SIR. It parallelises SIR and implements functionalities to use multiple random guesses.
There are three different modes selectable:
\begin{itemize}
\item {} 
\sphinxAtStartPar
1C : 1 Component Inversion

\item {} 
\sphinxAtStartPar
2C : 2 Component Inversion

\item {} 
\sphinxAtStartPar
MC : Monte\sphinxhyphen{}Carlo Simulation

\end{itemize}

\sphinxAtStartPar
It is also possible to use it without random guesses.

\sphinxAtStartPar
Check out the {\hyperref[\detokenize{usage::doc}]{\sphinxcrossref{\DUrole{doc}{Usage}}}} section for further information.

\sphinxAtStartPar
Check out the {\hyperref[\detokenize{functions::doc}]{\sphinxcrossref{\DUrole{doc}{Functions}}}} section for a list of functions.

\sphinxAtStartPar
Check out the {\hyperref[\detokenize{classes::doc}]{\sphinxcrossref{\DUrole{doc}{Classes}}}} section for a list of classes which are used to read and write the Stokes profiles and the models.

\begin{sphinxadmonition}{note}{Note:}
\sphinxAtStartPar
This project is under active development.
\end{sphinxadmonition}


\section{Contents}
\label{\detokenize{index:contents}}
\sphinxstepscope


\subsection{Usage}
\label{\detokenize{usage:usage}}\label{\detokenize{usage::doc}}

\subsubsection{Installation}
\label{\detokenize{usage:installation}}\label{\detokenize{usage:id1}}
\sphinxAtStartPar
To use SIR\sphinxhyphen{}MIG, first install the necessary libraries as defined in the text file requirements.py. The code might also work for other versions as mentioned there but it is not tested.

\sphinxAtStartPar
The necessary files with the right version can be installed by executing any of the two following lines in the directory of SIR\sphinxhyphen{}MIG:

\begin{sphinxVerbatim}[commandchars=\\\{\}]
\PYG{g+gp+gpVirtualEnv}{(.venv)} \PYG{g+gp}{\PYGZdl{} }pip\PYG{+w}{ }install\PYG{+w}{ }pipreqs
\PYG{g+gp+gpVirtualEnv}{(.venv)} \PYG{g+gp}{\PYGZdl{} }conda\PYG{+w}{ }install\PYG{+w}{ }\PYGZhy{}\PYGZhy{}yes\PYG{+w}{ }\PYGZhy{}\PYGZhy{}file\PYG{+w}{ }requirements.txt\PYG{+w}{ }\PYGZhy{}c\PYG{+w}{ }conda\PYGZhy{}forge
\end{sphinxVerbatim}

\sphinxstepscope


\subsection{Functions}
\label{\detokenize{functions:functions}}\label{\detokenize{functions::doc}}

\subsubsection{File sir.py}
\label{\detokenize{functions:file-sir-py}}
\sphinxAtStartPar
Library for repeating functions such as reading the config, writing SIR files.
\index{list\_to\_string() (in module sir)@\spxentry{list\_to\_string()}\spxextra{in module sir}}

\begin{fulllineitems}
\phantomsection\label{\detokenize{functions:sir.list_to_string}}
\pysigstartsignatures
\pysiglinewithargsret{\sphinxcode{\sphinxupquote{sir.}}\sphinxbfcode{\sphinxupquote{list\_to\_string}}}{\sphinxparam{\DUrole{n}{temp}}\sphinxparamcomma \sphinxparam{\DUrole{n}{let}\DUrole{o}{=}\DUrole{default_value}{\textquotesingle{},\textquotesingle{}}}}{}
\pysigstopsignatures
\sphinxAtStartPar
Convert a list to a string


\paragraph{Parameter}
\label{\detokenize{functions:parameter}}
\sphinxAtStartPar
temp : list
let : str
\begin{quote}

\sphinxAtStartPar
Letter which is added as a separation
\end{quote}


\paragraph{Return}
\label{\detokenize{functions:return}}
\sphinxAtStartPar
string with the information from the list

\end{fulllineitems}

\index{option() (in module sir)@\spxentry{option()}\spxextra{in module sir}}

\begin{fulllineitems}
\phantomsection\label{\detokenize{functions:sir.option}}
\pysigstartsignatures
\pysiglinewithargsret{\sphinxcode{\sphinxupquote{sir.}}\sphinxbfcode{\sphinxupquote{option}}}{\sphinxparam{\DUrole{n}{text1}}\sphinxparamcomma \sphinxparam{\DUrole{n}{text2}}}{}
\pysigstopsignatures
\sphinxAtStartPar
Print an option in a help page


\paragraph{Parameter}
\label{\detokenize{functions:id1}}\begin{description}
\sphinxlineitem{text1}{[}str{]}
\sphinxAtStartPar
First text

\sphinxlineitem{text2}{[}str{]}
\sphinxAtStartPar
Second text

\end{description}


\paragraph{Return}
\label{\detokenize{functions:id2}}
\sphinxAtStartPar
None

\end{fulllineitems}

\index{read\_chi2() (in module sir)@\spxentry{read\_chi2()}\spxextra{in module sir}}

\begin{fulllineitems}
\phantomsection\label{\detokenize{functions:sir.read_chi2}}
\pysigstartsignatures
\pysiglinewithargsret{\sphinxcode{\sphinxupquote{sir.}}\sphinxbfcode{\sphinxupquote{read\_chi2}}}{\sphinxparam{\DUrole{n}{filename}}\sphinxparamcomma \sphinxparam{\DUrole{n}{task}\DUrole{o}{=}\DUrole{default_value}{\textquotesingle{}\textquotesingle{}}}}{}
\pysigstopsignatures
\sphinxAtStartPar
Reads the last chi value in a inv.chi file


\paragraph{Parameter}
\label{\detokenize{functions:id3}}\begin{description}
\sphinxlineitem{filename}{[}string{]}
\sphinxAtStartPar
Path of the chi file

\sphinxlineitem{task}{[}string, optional{]}
\sphinxAtStartPar
Prints out in which folder the chi2 file does not exist. Default: ‘’

\end{description}


\paragraph{Return}
\label{\detokenize{functions:id4}}\begin{description}
\sphinxlineitem{chi2}{[}float{]}
\sphinxAtStartPar
Best chi2 value of the fit

\end{description}

\end{fulllineitems}

\index{read\_chi2s() (in module sir)@\spxentry{read\_chi2s()}\spxextra{in module sir}}

\begin{fulllineitems}
\phantomsection\label{\detokenize{functions:sir.read_chi2s}}
\pysigstartsignatures
\pysiglinewithargsret{\sphinxcode{\sphinxupquote{sir.}}\sphinxbfcode{\sphinxupquote{read\_chi2s}}}{\sphinxparam{\DUrole{n}{conf}}\sphinxparamcomma \sphinxparam{\DUrole{n}{tasks}}}{}
\pysigstopsignatures
\sphinxAtStartPar
Reads all the chi2 from the inversion


\paragraph{Parameter}
\label{\detokenize{functions:id5}}\begin{description}
\sphinxlineitem{config}{[}dict{]}
\sphinxAtStartPar
Config parameters

\sphinxlineitem{tasks}{[}dict{]}
\sphinxAtStartPar
Dictionary with the used folders

\end{description}


\paragraph{Return}
\label{\detokenize{functions:id6}}\begin{description}
\sphinxlineitem{chi2}{[}numpy array{]}
\sphinxAtStartPar
Numpy array with all chi2 values

\end{description}

\end{fulllineitems}

\index{read\_grid() (in module sir)@\spxentry{read\_grid()}\spxextra{in module sir}}

\begin{fulllineitems}
\phantomsection\label{\detokenize{functions:sir.read_grid}}
\pysigstartsignatures
\pysiglinewithargsret{\sphinxcode{\sphinxupquote{sir.}}\sphinxbfcode{\sphinxupquote{read\_grid}}}{\sphinxparam{\DUrole{n}{filename}}}{}
\pysigstopsignatures
\sphinxAtStartPar
Reads the grid file


\paragraph{Parameter}
\label{\detokenize{functions:id7}}\begin{description}
\sphinxlineitem{filename}{[}string{]}
\sphinxAtStartPar
File to be read

\end{description}


\paragraph{Return}
\label{\detokenize{functions:id8}}\begin{description}
\sphinxlineitem{dict}{[}Dictionary{]}
\sphinxAtStartPar
Dict. with ‘Line’, ‘min’, ‘step’ and ‘max’ in it

\end{description}

\end{fulllineitems}

\index{read\_line() (in module sir)@\spxentry{read\_line()}\spxextra{in module sir}}

\begin{fulllineitems}
\phantomsection\label{\detokenize{functions:sir.read_line}}
\pysigstartsignatures
\pysiglinewithargsret{\sphinxcode{\sphinxupquote{sir.}}\sphinxbfcode{\sphinxupquote{read\_line}}}{\sphinxparam{\DUrole{n}{filename}}}{}
\pysigstopsignatures
\sphinxAtStartPar
Reads the line file


\paragraph{Parameter}
\label{\detokenize{functions:id9}}\begin{description}
\sphinxlineitem{filename}{[}string{]}
\sphinxAtStartPar
File to be read

\end{description}


\paragraph{Return}
\label{\detokenize{functions:id10}}\begin{description}
\sphinxlineitem{dict}{[}Dictionary{]}
\sphinxAtStartPar
Dict. with ‘Line’, ‘Ion’, ‘wavelength’, ‘factor’, ‘Exc\_Pot’, log\_gf’,
‘Transition’, ‘alpha’ and ‘sigma’ in it

\end{description}

\end{fulllineitems}

\index{read\_config() (in module sir)@\spxentry{read\_config()}\spxextra{in module sir}}

\begin{fulllineitems}
\phantomsection\label{\detokenize{functions:sir.read_config}}
\pysigstartsignatures
\pysiglinewithargsret{\sphinxcode{\sphinxupquote{sir.}}\sphinxbfcode{\sphinxupquote{read\_config}}}{\sphinxparam{\DUrole{n}{filename}}\sphinxparamcomma \sphinxparam{\DUrole{n}{check}\DUrole{o}{=}\DUrole{default_value}{True}}\sphinxparamcomma \sphinxparam{\DUrole{n}{change\_config}\DUrole{o}{=}\DUrole{default_value}{False}}}{}
\pysigstopsignatures
\sphinxAtStartPar
Reads a config file for the inversion


\paragraph{Parameters}
\label{\detokenize{functions:parameters}}\begin{description}
\sphinxlineitem{filename}{[}string{]}
\sphinxAtStartPar
Path of the control file

\sphinxlineitem{check}{[}bool, optional{]}
\sphinxAtStartPar
Check if file exists (Default: True)

\sphinxlineitem{change\_config}{[}bool, optional{]}
\sphinxAtStartPar
config file is read to be changed (=\textgreater{} Do not try to load anything) (Default: False)

\end{description}


\paragraph{Returns}
\label{\detokenize{functions:returns}}\begin{description}
\sphinxlineitem{Dict}{[}dict{]}
\sphinxAtStartPar
Dict containing all the information from the config file

\end{description}

\end{fulllineitems}

\index{read\_control() (in module sir)@\spxentry{read\_control()}\spxextra{in module sir}}

\begin{fulllineitems}
\phantomsection\label{\detokenize{functions:sir.read_control}}
\pysigstartsignatures
\pysiglinewithargsret{\sphinxcode{\sphinxupquote{sir.}}\sphinxbfcode{\sphinxupquote{read\_control}}}{\sphinxparam{\DUrole{n}{filename}}}{}
\pysigstopsignatures
\sphinxAtStartPar
Reads a control file in the scheme SIR expects it.


\paragraph{Parameter}
\label{\detokenize{functions:id11}}\begin{description}
\sphinxlineitem{filename}{[}string{]}
\sphinxAtStartPar
Path of the control file

\end{description}


\paragraph{Return}
\label{\detokenize{functions:id12}}\begin{description}
\sphinxlineitem{Dict}{[}dict{]}
\sphinxAtStartPar
Dict containing all the information from the control file

\end{description}

\end{fulllineitems}

\index{read\_model() (in module sir)@\spxentry{read\_model()}\spxextra{in module sir}}

\begin{fulllineitems}
\phantomsection\label{\detokenize{functions:sir.read_model}}
\pysigstartsignatures
\pysiglinewithargsret{\sphinxcode{\sphinxupquote{sir.}}\sphinxbfcode{\sphinxupquote{read\_model}}}{\sphinxparam{\DUrole{n}{filename}}}{}
\pysigstopsignatures
\sphinxAtStartPar
Reads a model file and returns all parameters


\paragraph{Parameter}
\label{\detokenize{functions:id13}}\begin{description}
\sphinxlineitem{filename}{[}string{]}
\sphinxAtStartPar
String containing the path of the file

\end{description}


\paragraph{Return}
\label{\detokenize{functions:id14}}\begin{description}
\sphinxlineitem{log\_tau}{[}numpy.array{]}
\sphinxAtStartPar
Log tau

\sphinxlineitem{T}{[}numpy.array{]}
\sphinxAtStartPar
Temperature in K

\sphinxlineitem{Pe}{[}numpy.array{]}
\sphinxAtStartPar
Electron pressure in dyn/cm\textasciicircum{}2

\sphinxlineitem{v\_micro}{[}numpy.array{]}
\sphinxAtStartPar
Microturbulence velocity in cm/s

\sphinxlineitem{B}{[}numpy.array{]}
\sphinxAtStartPar
Magnetic field strength in Gauss

\sphinxlineitem{vlos}{[}numpy.array{]}
\sphinxAtStartPar
Line\sphinxhyphen{}of\sphinxhyphen{}sight velocity in cm/s

\sphinxlineitem{inc}{[}numpy.array{]}
\sphinxAtStartPar
Inclination in deg

\sphinxlineitem{azimuth}{[}numpy.array{]}
\sphinxAtStartPar
Azimuth angle in deg

\sphinxlineitem{z}{[}numpy.array, optional{]}
\sphinxAtStartPar
Height in km

\sphinxlineitem{Pg}{[}numpy.array, optional{]}
\sphinxAtStartPar
Gas pressure in dyn/cm\textasciicircum{}2

\sphinxlineitem{rho}{[}numpy.array, optional{]}
\sphinxAtStartPar
Density in g/cm\textasciicircum{}3

\end{description}

\end{fulllineitems}

\index{read\_profile() (in module sir)@\spxentry{read\_profile()}\spxextra{in module sir}}

\begin{fulllineitems}
\phantomsection\label{\detokenize{functions:sir.read_profile}}
\pysigstartsignatures
\pysiglinewithargsret{\sphinxcode{\sphinxupquote{sir.}}\sphinxbfcode{\sphinxupquote{read\_profile}}}{\sphinxparam{\DUrole{n}{filename}}\sphinxparamcomma \sphinxparam{\DUrole{n}{num}\DUrole{o}{=}\DUrole{default_value}{0}}}{}
\pysigstopsignatures
\sphinxAtStartPar
Reads the first LINE data from a profile computed by SIR


\paragraph{Parameter}
\label{\detokenize{functions:id15}}\begin{description}
\sphinxlineitem{filename}{[}string{]}
\sphinxAtStartPar
String containing the path of the file

\sphinxlineitem{num}{[}int, optional{]}
\sphinxAtStartPar
Number of the line which is loaded. Default: 0 (use first one from line)

\end{description}


\paragraph{Return}
\label{\detokenize{functions:id16}}\begin{description}
\sphinxlineitem{ll}{[}numpy.array{]}
\sphinxAtStartPar
Wavelengths in A

\sphinxlineitem{I}{[}numpy.array{]}
\sphinxAtStartPar
Stokes I

\sphinxlineitem{Q}{[}numpy.array{]}
\sphinxAtStartPar
Stokes Q

\sphinxlineitem{U}{[}numpy.array{]}
\sphinxAtStartPar
Stokes U

\sphinxlineitem{V}{[}numpy.array {]}
\sphinxAtStartPar
Stokes V

\end{description}

\end{fulllineitems}

\index{write\_config\_1c() (in module sir)@\spxentry{write\_config\_1c()}\spxextra{in module sir}}

\begin{fulllineitems}
\phantomsection\label{\detokenize{functions:sir.write_config_1c}}
\pysigstartsignatures
\pysiglinewithargsret{\sphinxcode{\sphinxupquote{sir.}}\sphinxbfcode{\sphinxupquote{write\_config\_1c}}}{\sphinxparam{\DUrole{n}{File}}\sphinxparamcomma \sphinxparam{\DUrole{n}{conf}}}{}
\pysigstopsignatures
\sphinxAtStartPar
Writes a config file with the information provided as a dictionary for the mode 1C


\paragraph{Parameters}
\label{\detokenize{functions:id17}}\begin{description}
\sphinxlineitem{File}{[}string{]}
\sphinxAtStartPar
Save path

\sphinxlineitem{conf}{[}dict{]}
\sphinxAtStartPar
Dictionary with all the informations

\end{description}

\end{fulllineitems}

\index{write\_config\_2c() (in module sir)@\spxentry{write\_config\_2c()}\spxextra{in module sir}}

\begin{fulllineitems}
\phantomsection\label{\detokenize{functions:sir.write_config_2c}}
\pysigstartsignatures
\pysiglinewithargsret{\sphinxcode{\sphinxupquote{sir.}}\sphinxbfcode{\sphinxupquote{write\_config\_2c}}}{\sphinxparam{\DUrole{n}{File}}\sphinxparamcomma \sphinxparam{\DUrole{n}{conf}}}{}
\pysigstopsignatures
\sphinxAtStartPar
Writes a config file with the information provided as a dictionary


\paragraph{Parameters}
\label{\detokenize{functions:id18}}\begin{description}
\sphinxlineitem{File}{[}string{]}
\sphinxAtStartPar
Save path

\sphinxlineitem{conf}{[}dict{]}
\sphinxAtStartPar
Dictionary with all the informations

\end{description}

\end{fulllineitems}

\index{write\_config\_mc() (in module sir)@\spxentry{write\_config\_mc()}\spxextra{in module sir}}

\begin{fulllineitems}
\phantomsection\label{\detokenize{functions:sir.write_config_mc}}
\pysigstartsignatures
\pysiglinewithargsret{\sphinxcode{\sphinxupquote{sir.}}\sphinxbfcode{\sphinxupquote{write\_config\_mc}}}{\sphinxparam{\DUrole{n}{File}}\sphinxparamcomma \sphinxparam{\DUrole{n}{conf}}}{}
\pysigstopsignatures
\sphinxAtStartPar
Writes a config file with the information provided as a dictionary


\paragraph{Parameters}
\label{\detokenize{functions:id19}}\begin{description}
\sphinxlineitem{File}{[}string{]}
\sphinxAtStartPar
Save path

\sphinxlineitem{conf}{[}dict{]}
\sphinxAtStartPar
Dictionary with all the informations

\end{description}

\end{fulllineitems}

\index{write\_config() (in module sir)@\spxentry{write\_config()}\spxextra{in module sir}}

\begin{fulllineitems}
\phantomsection\label{\detokenize{functions:sir.write_config}}
\pysigstartsignatures
\pysiglinewithargsret{\sphinxcode{\sphinxupquote{sir.}}\sphinxbfcode{\sphinxupquote{write\_config}}}{\sphinxparam{\DUrole{n}{File}}\sphinxparamcomma \sphinxparam{\DUrole{n}{conf}}}{}
\pysigstopsignatures
\sphinxAtStartPar
Writes a config file with the information provided as a dictionary for the mode 1C


\paragraph{Parameters}
\label{\detokenize{functions:id20}}\begin{description}
\sphinxlineitem{File}{[}string{]}
\sphinxAtStartPar
Save path

\sphinxlineitem{conf}{[}dict{]}
\sphinxAtStartPar
Dictionary with all the informations

\end{description}

\end{fulllineitems}

\index{write\_control\_1c() (in module sir)@\spxentry{write\_control\_1c()}\spxextra{in module sir}}

\begin{fulllineitems}
\phantomsection\label{\detokenize{functions:sir.write_control_1c}}
\pysigstartsignatures
\pysiglinewithargsret{\sphinxcode{\sphinxupquote{sir.}}\sphinxbfcode{\sphinxupquote{write\_control\_1c}}}{\sphinxparam{\DUrole{n}{filename}}\sphinxparamcomma \sphinxparam{\DUrole{n}{conf}}}{}
\pysigstopsignatures
\sphinxAtStartPar
Writes a control file in the scheme SIR expects it.


\paragraph{Parameter}
\label{\detokenize{functions:id21}}\begin{description}
\sphinxlineitem{filename}{[}string{]}
\sphinxAtStartPar
Save filename of the control file. Typically it is inv.trol

\sphinxlineitem{config}{[}dict{]}
\sphinxAtStartPar
Dictionary with the information from the config file

\end{description}

\end{fulllineitems}

\index{write\_control\_2c() (in module sir)@\spxentry{write\_control\_2c()}\spxextra{in module sir}}

\begin{fulllineitems}
\phantomsection\label{\detokenize{functions:sir.write_control_2c}}
\pysigstartsignatures
\pysiglinewithargsret{\sphinxcode{\sphinxupquote{sir.}}\sphinxbfcode{\sphinxupquote{write\_control\_2c}}}{\sphinxparam{\DUrole{n}{filename}}\sphinxparamcomma \sphinxparam{\DUrole{n}{conf}}}{}
\pysigstopsignatures
\sphinxAtStartPar
Writes a control file in the scheme SIR expects it.


\paragraph{Parameter}
\label{\detokenize{functions:id22}}\begin{description}
\sphinxlineitem{filename}{[}string{]}
\sphinxAtStartPar
Save filename of the control file. Typically it is inv.trol

\sphinxlineitem{conf}{[}dict{]}
\sphinxAtStartPar
Dictionary with the information from the config file

\end{description}

\end{fulllineitems}

\index{write\_control\_mc() (in module sir)@\spxentry{write\_control\_mc()}\spxextra{in module sir}}

\begin{fulllineitems}
\phantomsection\label{\detokenize{functions:sir.write_control_mc}}
\pysigstartsignatures
\pysiglinewithargsret{\sphinxcode{\sphinxupquote{sir.}}\sphinxbfcode{\sphinxupquote{write\_control\_mc}}}{\sphinxparam{\DUrole{n}{filename}}\sphinxparamcomma \sphinxparam{\DUrole{n}{conf}}\sphinxparamcomma \sphinxparam{\DUrole{n}{Type}\DUrole{o}{=}\DUrole{default_value}{\textquotesingle{}inv\textquotesingle{}}}}{}
\pysigstopsignatures
\sphinxAtStartPar
Writes a control file in the scheme SIR expects it.


\paragraph{Parameter}
\label{\detokenize{functions:id23}}\begin{description}
\sphinxlineitem{filename}{[}string{]}
\sphinxAtStartPar
Save filename of the control file. Typically it is inv.trol

\sphinxlineitem{config}{[}dict{]}
\sphinxAtStartPar
Dictionary with the information from the config file

\sphinxlineitem{Type}{[}string{]}
\sphinxAtStartPar
which type of control file is created (‘syn’ for synthesis, ‘inv’ for inversion)

\end{description}

\end{fulllineitems}

\index{write\_grid() (in module sir)@\spxentry{write\_grid()}\spxextra{in module sir}}

\begin{fulllineitems}
\phantomsection\label{\detokenize{functions:sir.write_grid}}
\pysigstartsignatures
\pysiglinewithargsret{\sphinxcode{\sphinxupquote{sir.}}\sphinxbfcode{\sphinxupquote{write\_grid}}}{\sphinxparam{\DUrole{n}{conf}}\sphinxparamcomma \sphinxparam{\DUrole{n}{waves}}\sphinxparamcomma \sphinxparam{\DUrole{n}{filename}\DUrole{o}{=}\DUrole{default_value}{\textquotesingle{}Grid.grid\textquotesingle{}}}}{}
\pysigstopsignatures
\sphinxAtStartPar
Writes the Grid file with data from the config file


\paragraph{Parameter}
\label{\detokenize{functions:id24}}\begin{description}
\sphinxlineitem{config}{[}dict{]}
\sphinxAtStartPar
Dictionary containing all the information from the config file

\sphinxlineitem{filename}{[}string, optional{]}
\sphinxAtStartPar
String containing the name of the Grid file. Default: Grid.grid

\end{description}

\end{fulllineitems}

\index{write\_grid\_mc() (in module sir)@\spxentry{write\_grid\_mc()}\spxextra{in module sir}}

\begin{fulllineitems}
\phantomsection\label{\detokenize{functions:sir.write_grid_mc}}
\pysigstartsignatures
\pysiglinewithargsret{\sphinxcode{\sphinxupquote{sir.}}\sphinxbfcode{\sphinxupquote{write\_grid\_mc}}}{\sphinxparam{\DUrole{n}{conf}}\sphinxparamcomma \sphinxparam{\DUrole{n}{filename}\DUrole{o}{=}\DUrole{default_value}{\textquotesingle{}Grid.grid\textquotesingle{}}}}{}
\pysigstopsignatures
\sphinxAtStartPar
Writes the Grid file with data from the config file


\paragraph{Parameter}
\label{\detokenize{functions:id25}}\begin{description}
\sphinxlineitem{config}{[}dict{]}
\sphinxAtStartPar
Dictionary containing all the information from the config file

\sphinxlineitem{filename}{[}string, optional{]}
\sphinxAtStartPar
String containing the name of the Grid file. Default: Grid.grid

\end{description}

\end{fulllineitems}

\index{write\_model() (in module sir)@\spxentry{write\_model()}\spxextra{in module sir}}

\begin{fulllineitems}
\phantomsection\label{\detokenize{functions:sir.write_model}}
\pysigstartsignatures
\pysiglinewithargsret{\sphinxcode{\sphinxupquote{sir.}}\sphinxbfcode{\sphinxupquote{write\_model}}}{\sphinxparam{\DUrole{n}{filename}}\sphinxparamcomma \sphinxparam{\DUrole{n}{Header}}\sphinxparamcomma \sphinxparam{\DUrole{n}{log\_tau}}\sphinxparamcomma \sphinxparam{\DUrole{n}{T}}\sphinxparamcomma \sphinxparam{\DUrole{n}{Pe}}\sphinxparamcomma \sphinxparam{\DUrole{n}{v\_micro}}\sphinxparamcomma \sphinxparam{\DUrole{n}{B}}\sphinxparamcomma \sphinxparam{\DUrole{n}{vlos}}\sphinxparamcomma \sphinxparam{\DUrole{n}{inc}}\sphinxparamcomma \sphinxparam{\DUrole{n}{azimuth}}\sphinxparamcomma \sphinxparam{\DUrole{n}{z}\DUrole{o}{=}\DUrole{default_value}{None}}\sphinxparamcomma \sphinxparam{\DUrole{n}{Pg}\DUrole{o}{=}\DUrole{default_value}{None}}\sphinxparamcomma \sphinxparam{\DUrole{n}{rho}\DUrole{o}{=}\DUrole{default_value}{None}}}{}
\pysigstopsignatures
\sphinxAtStartPar
Write a model with the given data in a specific format. Note that negative values
have one white space less


\paragraph{Parameter}
\label{\detokenize{functions:id26}}\begin{description}
\sphinxlineitem{filename}{[}string{]}
\sphinxAtStartPar
Name of the saved file

\sphinxlineitem{Header}{[}string{]}
\sphinxAtStartPar
Header of the model

\sphinxlineitem{log\_tau}{[}numpy.array{]}
\sphinxAtStartPar
Log tau

\sphinxlineitem{T}{[}numpy.array{]}
\sphinxAtStartPar
Temperature in K

\sphinxlineitem{Pe}{[}numpy.array{]}
\sphinxAtStartPar
Electron pressure in dyn/cm\textasciicircum{}2

\sphinxlineitem{v\_micro}{[}numpy.array{]}
\sphinxAtStartPar
Microturbulence velocity in cm/s

\sphinxlineitem{B}{[}numpy.array{]}
\sphinxAtStartPar
Magnetic field strength in Gauss

\sphinxlineitem{vlos}{[}numpy.array{]}
\sphinxAtStartPar
Line\sphinxhyphen{}of\sphinxhyphen{}sight velocity in cm/s

\sphinxlineitem{inc}{[}numpy.array{]}
\sphinxAtStartPar
Inclination in deg

\sphinxlineitem{azimuth}{[}numpy.array{]}
\sphinxAtStartPar
Azimuth angle in deg

\sphinxlineitem{z}{[}numpy.array, optional{]}
\sphinxAtStartPar
Height in km

\sphinxlineitem{Pg}{[}numpy.array, optional{]}
\sphinxAtStartPar
Gas pressure in dyn/cm\textasciicircum{}2

\sphinxlineitem{rho}{[}numpy.array, optional{]}
\sphinxAtStartPar
Density in g/cm\textasciicircum{}3

\end{description}


\paragraph{Return}
\label{\detokenize{functions:id27}}
\sphinxAtStartPar
None

\end{fulllineitems}

\index{write\_profile() (in module sir)@\spxentry{write\_profile()}\spxextra{in module sir}}

\begin{fulllineitems}
\phantomsection\label{\detokenize{functions:sir.write_profile}}
\pysigstartsignatures
\pysiglinewithargsret{\sphinxcode{\sphinxupquote{sir.}}\sphinxbfcode{\sphinxupquote{write\_profile}}}{\sphinxparam{\DUrole{n}{filename}}\sphinxparamcomma \sphinxparam{\DUrole{n}{profiles}}\sphinxparamcomma \sphinxparam{\DUrole{n}{pos}}}{}
\pysigstopsignatures
\sphinxAtStartPar
Write a profile for a specific model number to a file


\paragraph{Parameter}
\label{\detokenize{functions:id28}}\begin{description}
\sphinxlineitem{filename}{[}string{]}
\sphinxAtStartPar
Name of the saved file

\sphinxlineitem{profiles}{[}list{]}
\sphinxAtStartPar
List containing all the profiles

\sphinxlineitem{atoms}{[}list{]}
\sphinxAtStartPar
List containing the number of the line from the Line file

\sphinxlineitem{pos}{[}int{]}
\sphinxAtStartPar
Position which model is saved

\end{description}


\paragraph{Return}
\label{\detokenize{functions:id29}}
\sphinxAtStartPar
None

\end{fulllineitems}



\subsubsection{File create\_config.py}
\label{\detokenize{functions:file-create-config-py}}
\sphinxAtStartPar
Functions to create the config files for the different modes.
\index{config\_MC() (in module create\_config)@\spxentry{config\_MC()}\spxextra{in module create\_config}}

\begin{fulllineitems}
\phantomsection\label{\detokenize{functions:create_config.config_MC}}
\pysigstartsignatures
\pysiglinewithargsret{\sphinxcode{\sphinxupquote{create\_config.}}\sphinxbfcode{\sphinxupquote{config\_MC}}}{}{}
\pysigstopsignatures
\sphinxAtStartPar
Creates a config file for the Monte\sphinxhyphen{}Carlo simulation by asking questions.


\paragraph{Parameters}
\label{\detokenize{functions:id30}}
\sphinxAtStartPar
None


\paragraph{Returns}
\label{\detokenize{functions:id31}}
\sphinxAtStartPar
None

\end{fulllineitems}

\index{config\_1C() (in module create\_config)@\spxentry{config\_1C()}\spxextra{in module create\_config}}

\begin{fulllineitems}
\phantomsection\label{\detokenize{functions:create_config.config_1C}}
\pysigstartsignatures
\pysiglinewithargsret{\sphinxcode{\sphinxupquote{create\_config.}}\sphinxbfcode{\sphinxupquote{config\_1C}}}{}{}
\pysigstopsignatures
\sphinxAtStartPar
Creates a config file for the 1 Component Inversion by asking questions.


\paragraph{Parameters}
\label{\detokenize{functions:id32}}
\sphinxAtStartPar
None


\paragraph{Returns}
\label{\detokenize{functions:id33}}
\sphinxAtStartPar
None

\end{fulllineitems}

\index{config\_2C() (in module create\_config)@\spxentry{config\_2C()}\spxextra{in module create\_config}}

\begin{fulllineitems}
\phantomsection\label{\detokenize{functions:create_config.config_2C}}
\pysigstartsignatures
\pysiglinewithargsret{\sphinxcode{\sphinxupquote{create\_config.}}\sphinxbfcode{\sphinxupquote{config\_2C}}}{}{}
\pysigstopsignatures
\sphinxAtStartPar
Creates a config file for the 2 Components Inversion by asking questions.


\paragraph{Parameters}
\label{\detokenize{functions:id34}}
\sphinxAtStartPar
None


\paragraph{Returns}
\label{\detokenize{functions:id35}}
\sphinxAtStartPar
None

\end{fulllineitems}



\subsubsection{File main.py}
\label{\detokenize{functions:file-main-py}}
\sphinxAtStartPar
Main file to start the program
\index{main() (in module main)@\spxentry{main()}\spxextra{in module main}}

\begin{fulllineitems}
\phantomsection\label{\detokenize{functions:main.main}}
\pysigstartsignatures
\pysiglinewithargsret{\sphinxcode{\sphinxupquote{main.}}\sphinxbfcode{\sphinxupquote{main}}}{}{}
\pysigstopsignatures
\sphinxAtStartPar
Function which executes the programme


\paragraph{Parameters}
\label{\detokenize{functions:id36}}
\sphinxAtStartPar
None


\paragraph{Returns}
\label{\detokenize{functions:id37}}
\sphinxAtStartPar
None

\end{fulllineitems}



\subsubsection{File model.py}
\label{\detokenize{functions:file-model-py}}
\sphinxAtStartPar
Function to read a binary model file with all the physical parameter
\index{read\_model() (in module model)@\spxentry{read\_model()}\spxextra{in module model}}

\begin{fulllineitems}
\phantomsection\label{\detokenize{functions:model.read_model}}
\pysigstartsignatures
\pysiglinewithargsret{\sphinxcode{\sphinxupquote{model.}}\sphinxbfcode{\sphinxupquote{read\_model}}}{\sphinxparam{\DUrole{n}{filename}}}{}
\pysigstopsignatures
\end{fulllineitems}



\subsubsection{File misc.py}
\label{\detokenize{functions:file-misc-py}}
\sphinxAtStartPar
Miscellaneous functions
.. autofunction:: misc.initial


\subsubsection{File obs.py}
\label{\detokenize{functions:file-obs-py}}
\sphinxAtStartPar
Functions related to the observation
\index{read\_profile() (in module obs)@\spxentry{read\_profile()}\spxextra{in module obs}}

\begin{fulllineitems}
\phantomsection\label{\detokenize{functions:obs.read_profile}}
\pysigstartsignatures
\pysiglinewithargsret{\sphinxcode{\sphinxupquote{obs.}}\sphinxbfcode{\sphinxupquote{read\_profile}}}{\sphinxparam{\DUrole{n}{profile}}\sphinxparamcomma \sphinxparam{\DUrole{n}{grid}}\sphinxparamcomma \sphinxparam{\DUrole{n}{line\_file}}\sphinxparamcomma \sphinxparam{\DUrole{n}{waves}}}{}
\pysigstopsignatures
\end{fulllineitems}

\index{write\_psf() (in module obs)@\spxentry{write\_psf()}\spxextra{in module obs}}

\begin{fulllineitems}
\phantomsection\label{\detokenize{functions:obs.write_psf}}
\pysigstartsignatures
\pysiglinewithargsret{\sphinxcode{\sphinxupquote{obs.}}\sphinxbfcode{\sphinxupquote{write\_psf}}}{\sphinxparam{\DUrole{n}{conf}}\sphinxparamcomma \sphinxparam{\DUrole{n}{filename}}}{}
\pysigstopsignatures
\sphinxAtStartPar
Writes the spectral point spread function with the value from the spectral veil correction


\paragraph{Parameter}
\label{\detokenize{functions:id38}}\begin{description}
\sphinxlineitem{config}{[}dict{]}
\sphinxAtStartPar
Dictionary containing all the information of the config file

\sphinxlineitem{filename}{[}string{]}
\sphinxAtStartPar
Filename under which the file is saved

\end{description}


\paragraph{Return}
\label{\detokenize{functions:id39}}
\sphinxAtStartPar
None

\end{fulllineitems}



\subsubsection{File profile\_stk.py}
\label{\detokenize{functions:file-profile-stk-py}}
\sphinxAtStartPar
Function to read a binary profile file with the four Stokes Parameter

\sphinxstepscope


\subsection{Classes}
\label{\detokenize{classes:classes}}\label{\detokenize{classes::doc}}

\subsubsection{File model.py}
\label{\detokenize{classes:file-model-py}}
\sphinxAtStartPar
Class model with all the physical parameter
\index{Model (class in model)@\spxentry{Model}\spxextra{class in model}}

\begin{fulllineitems}
\phantomsection\label{\detokenize{classes:model.Model}}
\pysigstartsignatures
\pysiglinewithargsret{\sphinxbfcode{\sphinxupquote{class\DUrole{w}{ }}}\sphinxcode{\sphinxupquote{model.}}\sphinxbfcode{\sphinxupquote{Model}}}{\sphinxparam{\DUrole{n}{nx}\DUrole{o}{=}\DUrole{default_value}{0}}\sphinxparamcomma \sphinxparam{\DUrole{n}{ny}\DUrole{o}{=}\DUrole{default_value}{0}}\sphinxparamcomma \sphinxparam{\DUrole{n}{nval}\DUrole{o}{=}\DUrole{default_value}{0}}}{}
\pysigstopsignatures
\sphinxAtStartPar
Class containing the models of a simulation
\begin{description}
\sphinxlineitem{Variables are:}\begin{itemize}
\item {} 
\sphinxAtStartPar
log\_tau

\item {} 
\sphinxAtStartPar
T

\item {} 
\sphinxAtStartPar
Pe

\item {} 
\sphinxAtStartPar
vmicro

\item {} 
\sphinxAtStartPar
B

\item {} 
\sphinxAtStartPar
vlos

\item {} 
\sphinxAtStartPar
gamma

\item {} 
\sphinxAtStartPar
phi

\item {} 
\sphinxAtStartPar
z

\item {} 
\sphinxAtStartPar
pg

\item {} 
\sphinxAtStartPar
rho

\end{itemize}

\sphinxlineitem{There are several functions:}\begin{itemize}
\item {} 
\sphinxAtStartPar
read: Read a numpy file containing models and stores it into the class variables

\item {} 
\sphinxAtStartPar
read\_results: Reads the results from the inversion of SIR

\item {} 
\sphinxAtStartPar
write: Writes a Model to a SIR readable file

\item {} 
\sphinxAtStartPar
save: Saves the models as a numpy file

\item {} 
\sphinxAtStartPar
set\_limit: Cuts the data to a specific log\_tau value (used for plotting)

\end{itemize}

\end{description}
\index{\_\_init\_\_() (model.Model method)@\spxentry{\_\_init\_\_()}\spxextra{model.Model method}}

\begin{fulllineitems}
\phantomsection\label{\detokenize{classes:model.Model.__init__}}
\pysigstartsignatures
\pysiglinewithargsret{\sphinxbfcode{\sphinxupquote{\_\_init\_\_}}}{\sphinxparam{\DUrole{n}{nx}\DUrole{o}{=}\DUrole{default_value}{0}}\sphinxparamcomma \sphinxparam{\DUrole{n}{ny}\DUrole{o}{=}\DUrole{default_value}{0}}\sphinxparamcomma \sphinxparam{\DUrole{n}{nval}\DUrole{o}{=}\DUrole{default_value}{0}}}{}
\pysigstopsignatures
\sphinxAtStartPar
Initialisation of the class with the models


\paragraph{Parameter}
\label{\detokenize{classes:parameter}}\begin{description}
\sphinxlineitem{filename}{[}str, optional{]}
\sphinxAtStartPar
File name of the model to be loaded. Default: None (= no reading)

\sphinxlineitem{filename\_fill}{[}str, optional{]}
\sphinxAtStartPar
File name of the filling factor

\end{description}

\end{fulllineitems}

\index{correct\_phi() (model.Model method)@\spxentry{correct\_phi()}\spxextra{model.Model method}}

\begin{fulllineitems}
\phantomsection\label{\detokenize{classes:model.Model.correct_phi}}
\pysigstartsignatures
\pysiglinewithargsret{\sphinxbfcode{\sphinxupquote{correct\_phi}}}{}{}
\pysigstopsignatures
\sphinxAtStartPar
SIR can give you any value for the azimuth, e.g. \sphinxhyphen{}250 deg, and therefore, to compute the standard deviation
those values should be corrected so that I have values from 0 to 360 degrees.


\paragraph{Parameter}
\label{\detokenize{classes:id1}}
\sphinxAtStartPar
None


\paragraph{Return}
\label{\detokenize{classes:return}}
\sphinxAtStartPar
class

\end{fulllineitems}

\index{cut\_to\_map() (model.Model method)@\spxentry{cut\_to\_map()}\spxextra{model.Model method}}

\begin{fulllineitems}
\phantomsection\label{\detokenize{classes:model.Model.cut_to_map}}
\pysigstartsignatures
\pysiglinewithargsret{\sphinxbfcode{\sphinxupquote{cut\_to\_map}}}{\sphinxparam{\DUrole{n}{Map}}}{}
\pysigstopsignatures
\sphinxAtStartPar
Cut the data to a map {[}xmin, xmax, ymin, ymax{]}


\paragraph{Parameters}
\label{\detokenize{classes:parameters}}\begin{description}
\sphinxlineitem{Map}{[}list{]}
\sphinxAtStartPar
List with the ranges in pixel in x and y direction

\end{description}

\end{fulllineitems}

\index{get\_attribute() (model.Model method)@\spxentry{get\_attribute()}\spxextra{model.Model method}}

\begin{fulllineitems}
\phantomsection\label{\detokenize{classes:model.Model.get_attribute}}
\pysigstartsignatures
\pysiglinewithargsret{\sphinxbfcode{\sphinxupquote{get\_attribute}}}{\sphinxparam{\DUrole{n}{string}}}{}
\pysigstopsignatures
\sphinxAtStartPar
Returns a specific physical parameter. This can be used if ones only wants a specific
parameter depending on a string/input


\paragraph{Parameter}
\label{\detokenize{classes:id2}}\begin{description}
\sphinxlineitem{string}{[}str{]}
\sphinxAtStartPar
Determines which physical parameter is returned
Options are: tau, T, Pe, vmicro, B, vlos, gamma, phi, z, Pg, rho, nx, ny, npar, fill

\end{description}


\paragraph{Return}
\label{\detokenize{classes:id3}}
\sphinxAtStartPar
The wished physical parameter

\end{fulllineitems}

\index{read() (model.Model method)@\spxentry{read()}\spxextra{model.Model method}}

\begin{fulllineitems}
\phantomsection\label{\detokenize{classes:model.Model.read}}
\pysigstartsignatures
\pysiglinewithargsret{\sphinxbfcode{\sphinxupquote{read}}}{\sphinxparam{\DUrole{n}{fname}}\sphinxparamcomma \sphinxparam{\DUrole{n}{fmt\_type=\textless{}class \textquotesingle{}numpy.float64\textquotesingle{}\textgreater{}}}}{}
\pysigstopsignatures
\end{fulllineitems}

\index{read\_results() (model.Model method)@\spxentry{read\_results()}\spxextra{model.Model method}}

\begin{fulllineitems}
\phantomsection\label{\detokenize{classes:model.Model.read_results}}
\pysigstartsignatures
\pysiglinewithargsret{\sphinxbfcode{\sphinxupquote{read\_results}}}{\sphinxparam{\DUrole{n}{task}}\sphinxparamcomma \sphinxparam{\DUrole{n}{filename}}\sphinxparamcomma \sphinxparam{\DUrole{n}{path}}\sphinxparamcomma \sphinxparam{\DUrole{n}{nx}}\sphinxparamcomma \sphinxparam{\DUrole{n}{ny}}}{}
\pysigstopsignatures
\sphinxAtStartPar
Reads all the errors from the inversion


\paragraph{Parameter}
\label{\detokenize{classes:id4}}\begin{description}
\sphinxlineitem{task}{[}dict{]}
\sphinxAtStartPar
Dictionary with all the task folders, x and y positions

\sphinxlineitem{filename}{[}str{]}
\sphinxAtStartPar
Filename of the file in each task folder

\sphinxlineitem{path}{[}str{]}
\sphinxAtStartPar
Path where all the files are

\sphinxlineitem{nx}{[}int{]}
\sphinxAtStartPar
how many results are read in x

\sphinxlineitem{ny}{[}int{]}
\sphinxAtStartPar
how many results are read in y

\end{description}


\paragraph{Return}
\label{\detokenize{classes:id5}}
\sphinxAtStartPar
class with the parameters

\end{fulllineitems}

\index{set\_dim() (model.Model method)@\spxentry{set\_dim()}\spxextra{model.Model method}}

\begin{fulllineitems}
\phantomsection\label{\detokenize{classes:model.Model.set_dim}}
\pysigstartsignatures
\pysiglinewithargsret{\sphinxbfcode{\sphinxupquote{set\_dim}}}{\sphinxparam{\DUrole{n}{nx}}\sphinxparamcomma \sphinxparam{\DUrole{n}{ny}}\sphinxparamcomma \sphinxparam{\DUrole{n}{npars}}}{}
\pysigstopsignatures
\sphinxAtStartPar
Sets the dimensions if no data is loaded yet


\paragraph{Parameter}
\label{\detokenize{classes:id6}}\begin{description}
\sphinxlineitem{nx}{[}int{]}
\sphinxAtStartPar
Number of Models in x

\sphinxlineitem{ny}{[}int{]}
\sphinxAtStartPar
Number of Models in y

\sphinxlineitem{npars}{[}int{]}
\sphinxAtStartPar
Number of values per physical parameter

\end{description}

\end{fulllineitems}

\index{set\_limit() (model.Model method)@\spxentry{set\_limit()}\spxextra{model.Model method}}

\begin{fulllineitems}
\phantomsection\label{\detokenize{classes:model.Model.set_limit}}
\pysigstartsignatures
\pysiglinewithargsret{\sphinxbfcode{\sphinxupquote{set\_limit}}}{\sphinxparam{\DUrole{n}{lim}}}{}
\pysigstopsignatures
\sphinxAtStartPar
Cuts the arrays to a specific log tau value (should only be used for plotting)


\paragraph{Parameter}
\label{\detokenize{classes:id7}}\begin{description}
\sphinxlineitem{lim}{[}int{]}
\sphinxAtStartPar
log tau value to which the values are cut

\end{description}

\end{fulllineitems}

\index{write() (model.Model method)@\spxentry{write()}\spxextra{model.Model method}}

\begin{fulllineitems}
\phantomsection\label{\detokenize{classes:model.Model.write}}
\pysigstartsignatures
\pysiglinewithargsret{\sphinxbfcode{\sphinxupquote{write}}}{\sphinxparam{\DUrole{n}{fname}}\sphinxparamcomma \sphinxparam{\DUrole{n}{fmt\_type=\textless{}class \textquotesingle{}numpy.float64\textquotesingle{}\textgreater{}}}}{}
\pysigstopsignatures
\sphinxAtStartPar
Write data into a binary fortran file


\paragraph{Parameter}
\label{\detokenize{classes:id8}}\begin{description}
\sphinxlineitem{fname}{[}str{]}
\sphinxAtStartPar
File name

\sphinxlineitem{fmt\_type}{[}type{]}
\sphinxAtStartPar
type which is used to save it =\textgreater{} numpy.float64 used

\end{description}

\end{fulllineitems}

\index{write\_model() (model.Model method)@\spxentry{write\_model()}\spxextra{model.Model method}}

\begin{fulllineitems}
\phantomsection\label{\detokenize{classes:model.Model.write_model}}
\pysigstartsignatures
\pysiglinewithargsret{\sphinxbfcode{\sphinxupquote{write\_model}}}{\sphinxparam{\DUrole{n}{filename}}\sphinxparamcomma \sphinxparam{\DUrole{n}{header}}\sphinxparamcomma \sphinxparam{\DUrole{n}{x}}\sphinxparamcomma \sphinxparam{\DUrole{n}{y}}}{}
\pysigstopsignatures
\sphinxAtStartPar
Write a model with the given data in a specific format.


\paragraph{Parameter}
\label{\detokenize{classes:id9}}\begin{description}
\sphinxlineitem{filename}{[}string{]}
\sphinxAtStartPar
Name of the saved file

\sphinxlineitem{Header}{[}string{]}
\sphinxAtStartPar
Header of the model

\sphinxlineitem{x}{[}int{]}
\sphinxAtStartPar
Integer determining which model

\sphinxlineitem{y}{[}int{]}
\sphinxAtStartPar
Integer determining which model

\end{description}


\paragraph{Return}
\label{\detokenize{classes:id10}}
\sphinxAtStartPar
None

\end{fulllineitems}

\index{correct\_phi() (model.Model method)@\spxentry{correct\_phi()}\spxextra{model.Model method}}

\begin{fulllineitems}
\phantomsection\label{\detokenize{classes:id0}}
\pysigstartsignatures
\pysiglinewithargsret{\sphinxbfcode{\sphinxupquote{correct\_phi}}}{}{}
\pysigstopsignatures
\sphinxAtStartPar
SIR can give you any value for the azimuth, e.g. \sphinxhyphen{}250 deg, and therefore, to compute the standard deviation
those values should be corrected so that I have values from 0 to 360 degrees.


\paragraph{Parameter}
\label{\detokenize{classes:id11}}
\sphinxAtStartPar
None


\paragraph{Return}
\label{\detokenize{classes:id12}}
\sphinxAtStartPar
class

\end{fulllineitems}

\index{cut\_to\_map() (model.Model method)@\spxentry{cut\_to\_map()}\spxextra{model.Model method}}

\begin{fulllineitems}
\phantomsection\label{\detokenize{classes:id13}}
\pysigstartsignatures
\pysiglinewithargsret{\sphinxbfcode{\sphinxupquote{cut\_to\_map}}}{\sphinxparam{\DUrole{n}{Map}}}{}
\pysigstopsignatures
\sphinxAtStartPar
Cut the data to a map {[}xmin, xmax, ymin, ymax{]}


\paragraph{Parameters}
\label{\detokenize{classes:id14}}\begin{description}
\sphinxlineitem{Map}{[}list{]}
\sphinxAtStartPar
List with the ranges in pixel in x and y direction

\end{description}

\end{fulllineitems}

\index{get\_attribute() (model.Model method)@\spxentry{get\_attribute()}\spxextra{model.Model method}}

\begin{fulllineitems}
\phantomsection\label{\detokenize{classes:id15}}
\pysigstartsignatures
\pysiglinewithargsret{\sphinxbfcode{\sphinxupquote{get\_attribute}}}{\sphinxparam{\DUrole{n}{string}}}{}
\pysigstopsignatures
\sphinxAtStartPar
Returns a specific physical parameter. This can be used if ones only wants a specific
parameter depending on a string/input


\paragraph{Parameter}
\label{\detokenize{classes:id16}}\begin{description}
\sphinxlineitem{string}{[}str{]}
\sphinxAtStartPar
Determines which physical parameter is returned
Options are: tau, T, Pe, vmicro, B, vlos, gamma, phi, z, Pg, rho, nx, ny, npar, fill

\end{description}


\paragraph{Return}
\label{\detokenize{classes:id17}}
\sphinxAtStartPar
The wished physical parameter

\end{fulllineitems}

\index{read\_results() (model.Model method)@\spxentry{read\_results()}\spxextra{model.Model method}}

\begin{fulllineitems}
\phantomsection\label{\detokenize{classes:id18}}
\pysigstartsignatures
\pysiglinewithargsret{\sphinxbfcode{\sphinxupquote{read\_results}}}{\sphinxparam{\DUrole{n}{task}}\sphinxparamcomma \sphinxparam{\DUrole{n}{filename}}\sphinxparamcomma \sphinxparam{\DUrole{n}{path}}\sphinxparamcomma \sphinxparam{\DUrole{n}{nx}}\sphinxparamcomma \sphinxparam{\DUrole{n}{ny}}}{}
\pysigstopsignatures
\sphinxAtStartPar
Reads all the errors from the inversion


\paragraph{Parameter}
\label{\detokenize{classes:id19}}\begin{description}
\sphinxlineitem{task}{[}dict{]}
\sphinxAtStartPar
Dictionary with all the task folders, x and y positions

\sphinxlineitem{filename}{[}str{]}
\sphinxAtStartPar
Filename of the file in each task folder

\sphinxlineitem{path}{[}str{]}
\sphinxAtStartPar
Path where all the files are

\sphinxlineitem{nx}{[}int{]}
\sphinxAtStartPar
how many results are read in x

\sphinxlineitem{ny}{[}int{]}
\sphinxAtStartPar
how many results are read in y

\end{description}


\paragraph{Return}
\label{\detokenize{classes:id20}}
\sphinxAtStartPar
class with the parameters

\end{fulllineitems}

\index{set\_dim() (model.Model method)@\spxentry{set\_dim()}\spxextra{model.Model method}}

\begin{fulllineitems}
\phantomsection\label{\detokenize{classes:id21}}
\pysigstartsignatures
\pysiglinewithargsret{\sphinxbfcode{\sphinxupquote{set\_dim}}}{\sphinxparam{\DUrole{n}{nx}}\sphinxparamcomma \sphinxparam{\DUrole{n}{ny}}\sphinxparamcomma \sphinxparam{\DUrole{n}{npars}}}{}
\pysigstopsignatures
\sphinxAtStartPar
Sets the dimensions if no data is loaded yet


\paragraph{Parameter}
\label{\detokenize{classes:id22}}\begin{description}
\sphinxlineitem{nx}{[}int{]}
\sphinxAtStartPar
Number of Models in x

\sphinxlineitem{ny}{[}int{]}
\sphinxAtStartPar
Number of Models in y

\sphinxlineitem{npars}{[}int{]}
\sphinxAtStartPar
Number of values per physical parameter

\end{description}

\end{fulllineitems}

\index{set\_limit() (model.Model method)@\spxentry{set\_limit()}\spxextra{model.Model method}}

\begin{fulllineitems}
\phantomsection\label{\detokenize{classes:id23}}
\pysigstartsignatures
\pysiglinewithargsret{\sphinxbfcode{\sphinxupquote{set\_limit}}}{\sphinxparam{\DUrole{n}{lim}}}{}
\pysigstopsignatures
\sphinxAtStartPar
Cuts the arrays to a specific log tau value (should only be used for plotting)


\paragraph{Parameter}
\label{\detokenize{classes:id24}}\begin{description}
\sphinxlineitem{lim}{[}int{]}
\sphinxAtStartPar
log tau value to which the values are cut

\end{description}

\end{fulllineitems}

\index{write() (model.Model method)@\spxentry{write()}\spxextra{model.Model method}}

\begin{fulllineitems}
\phantomsection\label{\detokenize{classes:id25}}
\pysigstartsignatures
\pysiglinewithargsret{\sphinxbfcode{\sphinxupquote{write}}}{\sphinxparam{\DUrole{n}{fname}}\sphinxparamcomma \sphinxparam{\DUrole{n}{fmt\_type=\textless{}class \textquotesingle{}numpy.float64\textquotesingle{}\textgreater{}}}}{}
\pysigstopsignatures
\sphinxAtStartPar
Write data into a binary fortran file


\paragraph{Parameter}
\label{\detokenize{classes:id26}}\begin{description}
\sphinxlineitem{fname}{[}str{]}
\sphinxAtStartPar
File name

\sphinxlineitem{fmt\_type}{[}type{]}
\sphinxAtStartPar
type which is used to save it =\textgreater{} numpy.float64 used

\end{description}

\end{fulllineitems}

\index{write\_model() (model.Model method)@\spxentry{write\_model()}\spxextra{model.Model method}}

\begin{fulllineitems}
\phantomsection\label{\detokenize{classes:id27}}
\pysigstartsignatures
\pysiglinewithargsret{\sphinxbfcode{\sphinxupquote{write\_model}}}{\sphinxparam{\DUrole{n}{filename}}\sphinxparamcomma \sphinxparam{\DUrole{n}{header}}\sphinxparamcomma \sphinxparam{\DUrole{n}{x}}\sphinxparamcomma \sphinxparam{\DUrole{n}{y}}}{}
\pysigstopsignatures
\sphinxAtStartPar
Write a model with the given data in a specific format.


\paragraph{Parameter}
\label{\detokenize{classes:id28}}\begin{description}
\sphinxlineitem{filename}{[}string{]}
\sphinxAtStartPar
Name of the saved file

\sphinxlineitem{Header}{[}string{]}
\sphinxAtStartPar
Header of the model

\sphinxlineitem{x}{[}int{]}
\sphinxAtStartPar
Integer determining which model

\sphinxlineitem{y}{[}int{]}
\sphinxAtStartPar
Integer determining which model

\end{description}


\paragraph{Return}
\label{\detokenize{classes:id29}}
\sphinxAtStartPar
None

\end{fulllineitems}


\end{fulllineitems}



\subsubsection{File profile\_stk.py}
\label{\detokenize{classes:file-profile-stk-py}}
\sphinxAtStartPar
Class profile\_stk with the four Stokes Parameter
\index{Profile (class in profile\_stk)@\spxentry{Profile}\spxextra{class in profile\_stk}}

\begin{fulllineitems}
\phantomsection\label{\detokenize{classes:profile_stk.Profile}}
\pysigstartsignatures
\pysiglinewithargsret{\sphinxbfcode{\sphinxupquote{class\DUrole{w}{ }}}\sphinxcode{\sphinxupquote{profile\_stk.}}\sphinxbfcode{\sphinxupquote{Profile}}}{\sphinxparam{\DUrole{n}{nx}\DUrole{o}{=}\DUrole{default_value}{None}}\sphinxparamcomma \sphinxparam{\DUrole{n}{ny}\DUrole{o}{=}\DUrole{default_value}{None}}\sphinxparamcomma \sphinxparam{\DUrole{n}{nw}\DUrole{o}{=}\DUrole{default_value}{0}}}{}
\pysigstopsignatures
\sphinxAtStartPar
Class containing the models of a simulation
\begin{description}
\sphinxlineitem{Variables are:}\begin{itemize}
\item {} 
\sphinxAtStartPar
wave

\item {} 
\sphinxAtStartPar
stki

\item {} 
\sphinxAtStartPar
stkq

\item {} 
\sphinxAtStartPar
stku

\item {} 
\sphinxAtStartPar
stkv

\end{itemize}

\sphinxlineitem{There are several functions:}\begin{itemize}
\item {} 
\sphinxAtStartPar
read: Read a numpy file containing models and stores it into the class variables

\item {} 
\sphinxAtStartPar
read\_results: Reads the results from the inversion of SIR

\item {} 
\sphinxAtStartPar
write: Writes a Model to a SIR readable file

\item {} 
\sphinxAtStartPar
save: Saves the models as a numpy file

\item {} 
\sphinxAtStartPar
set\_limit: Cuts the data to a specific log\_tau value (used for plotting)

\end{itemize}

\end{description}
\index{\_\_init\_\_() (profile\_stk.Profile method)@\spxentry{\_\_init\_\_()}\spxextra{profile\_stk.Profile method}}

\begin{fulllineitems}
\phantomsection\label{\detokenize{classes:profile_stk.Profile.__init__}}
\pysigstartsignatures
\pysiglinewithargsret{\sphinxbfcode{\sphinxupquote{\_\_init\_\_}}}{\sphinxparam{\DUrole{n}{nx}\DUrole{o}{=}\DUrole{default_value}{None}}\sphinxparamcomma \sphinxparam{\DUrole{n}{ny}\DUrole{o}{=}\DUrole{default_value}{None}}\sphinxparamcomma \sphinxparam{\DUrole{n}{nw}\DUrole{o}{=}\DUrole{default_value}{0}}}{}
\pysigstopsignatures
\sphinxAtStartPar
Initialisation of the class with the Profiles


\paragraph{Parameter}
\label{\detokenize{classes:id30}}\begin{description}
\sphinxlineitem{nx}{[}int{]}
\sphinxAtStartPar
Integer of pixels in x direction

\sphinxlineitem{ny}{[}int{]}
\sphinxAtStartPar
Integer of pixels in y direction

\sphinxlineitem{nw}{[}int{]}
\sphinxAtStartPar
Number of wavelength points

\end{description}

\end{fulllineitems}

\index{\_\_read\_grid() (profile\_stk.Profile method)@\spxentry{\_\_read\_grid()}\spxextra{profile\_stk.Profile method}}

\begin{fulllineitems}
\phantomsection\label{\detokenize{classes:profile_stk.Profile.__read_grid}}
\pysigstartsignatures
\pysiglinewithargsret{\sphinxbfcode{\sphinxupquote{\_\_read\_grid}}}{}{}
\pysigstopsignatures
\sphinxAtStartPar
Reads the grid file


\paragraph{Parameter}
\label{\detokenize{classes:id31}}\begin{description}
\sphinxlineitem{filename}{[}string{]}
\sphinxAtStartPar
File to be read

\end{description}


\paragraph{Return}
\label{\detokenize{classes:id32}}\begin{description}
\sphinxlineitem{dict}{[}Dictionary{]}
\sphinxAtStartPar
Dict. with ‘Line’, ‘min’, ‘step’ and ‘max’ in it

\end{description}

\end{fulllineitems}

\index{\_\_read\_profile\_sir() (profile\_stk.Profile method)@\spxentry{\_\_read\_profile\_sir()}\spxextra{profile\_stk.Profile method}}

\begin{fulllineitems}
\phantomsection\label{\detokenize{classes:profile_stk.Profile.__read_profile_sir}}
\pysigstartsignatures
\pysiglinewithargsret{\sphinxbfcode{\sphinxupquote{\_\_read\_profile\_sir}}}{}{}
\pysigstopsignatures
\sphinxAtStartPar
Reads the first LINE data from a profile computed by SIR


\paragraph{Parameter}
\label{\detokenize{classes:id33}}\begin{description}
\sphinxlineitem{filename}{[}string{]}
\sphinxAtStartPar
String containing the path of the file

\end{description}


\paragraph{Return}
\label{\detokenize{classes:id34}}\begin{description}
\sphinxlineitem{ll}{[}numpy.array{]}
\sphinxAtStartPar
Wavelengths in A

\sphinxlineitem{I}{[}numpy.array{]}
\sphinxAtStartPar
Stokes I

\sphinxlineitem{Q}{[}numpy.array{]}
\sphinxAtStartPar
Stokes Q

\sphinxlineitem{U}{[}numpy.array{]}
\sphinxAtStartPar
Stokes U

\sphinxlineitem{V}{[}numpy.array {]}
\sphinxAtStartPar
Stokes V

\end{description}

\end{fulllineitems}

\index{\_\_read\_profile\_sir\_mc() (profile\_stk.Profile method)@\spxentry{\_\_read\_profile\_sir\_mc()}\spxextra{profile\_stk.Profile method}}

\begin{fulllineitems}
\phantomsection\label{\detokenize{classes:profile_stk.Profile.__read_profile_sir_mc}}
\pysigstartsignatures
\pysiglinewithargsret{\sphinxbfcode{\sphinxupquote{\_\_read\_profile\_sir\_mc}}}{}{}
\pysigstopsignatures
\sphinxAtStartPar
Reads the first LINE data from a profile computed by SIR


\paragraph{Parameter}
\label{\detokenize{classes:id35}}\begin{description}
\sphinxlineitem{filename}{[}string{]}
\sphinxAtStartPar
String containing the path of the file

\end{description}


\paragraph{Return}
\label{\detokenize{classes:id36}}\begin{description}
\sphinxlineitem{ll}{[}numpy.array{]}
\sphinxAtStartPar
Wavelengths in A

\sphinxlineitem{I}{[}numpy.array{]}
\sphinxAtStartPar
Stokes I

\sphinxlineitem{Q}{[}numpy.array{]}
\sphinxAtStartPar
Stokes Q

\sphinxlineitem{U}{[}numpy.array{]}
\sphinxAtStartPar
Stokes U

\sphinxlineitem{V}{[}numpy.array {]}
\sphinxAtStartPar
Stokes V

\end{description}

\end{fulllineitems}

\index{cut\_to\_map() (profile\_stk.Profile method)@\spxentry{cut\_to\_map()}\spxextra{profile\_stk.Profile method}}

\begin{fulllineitems}
\phantomsection\label{\detokenize{classes:profile_stk.Profile.cut_to_map}}
\pysigstartsignatures
\pysiglinewithargsret{\sphinxbfcode{\sphinxupquote{cut\_to\_map}}}{\sphinxparam{\DUrole{n}{Map}}}{}
\pysigstopsignatures
\sphinxAtStartPar
Cut the data to a map {[}xmin, xmax, ymin, ymax{]}


\paragraph{Parameters}
\label{\detokenize{classes:id37}}\begin{description}
\sphinxlineitem{Map}{[}list{]}
\sphinxAtStartPar
List with the ranges in pixel in x and y direction

\end{description}

\end{fulllineitems}

\index{cut\_to\_wave() (profile\_stk.Profile method)@\spxentry{cut\_to\_wave()}\spxextra{profile\_stk.Profile method}}

\begin{fulllineitems}
\phantomsection\label{\detokenize{classes:profile_stk.Profile.cut_to_wave}}
\pysigstartsignatures
\pysiglinewithargsret{\sphinxbfcode{\sphinxupquote{cut\_to\_wave}}}{\sphinxparam{\DUrole{n}{range\_wave}}}{}
\pysigstopsignatures
\sphinxAtStartPar
Cut the data to the range in wavelengths


\paragraph{Parameters}
\label{\detokenize{classes:id38}}\begin{description}
\sphinxlineitem{range\_wave}{[}list{]}
\sphinxAtStartPar
List with the ranges from the config file

\end{description}

\end{fulllineitems}

\index{read() (profile\_stk.Profile method)@\spxentry{read()}\spxextra{profile\_stk.Profile method}}

\begin{fulllineitems}
\phantomsection\label{\detokenize{classes:profile_stk.Profile.read}}
\pysigstartsignatures
\pysiglinewithargsret{\sphinxbfcode{\sphinxupquote{read}}}{\sphinxparam{\DUrole{n}{fname}}\sphinxparamcomma \sphinxparam{\DUrole{n}{fmt\_type=\textless{}class \textquotesingle{}numpy.float32\textquotesingle{}\textgreater{}}}}{}
\pysigstopsignatures
\end{fulllineitems}

\index{read\_results() (profile\_stk.Profile method)@\spxentry{read\_results()}\spxextra{profile\_stk.Profile method}}

\begin{fulllineitems}
\phantomsection\label{\detokenize{classes:profile_stk.Profile.read_results}}
\pysigstartsignatures
\pysiglinewithargsret{\sphinxbfcode{\sphinxupquote{read\_results}}}{\sphinxparam{\DUrole{n}{task}}\sphinxparamcomma \sphinxparam{\DUrole{n}{filename}}\sphinxparamcomma \sphinxparam{\DUrole{n}{path}}\sphinxparamcomma \sphinxparam{\DUrole{n}{nx}}\sphinxparamcomma \sphinxparam{\DUrole{n}{ny}}}{}
\pysigstopsignatures
\sphinxAtStartPar
Reads all the errors from the inversion


\paragraph{Parameter}
\label{\detokenize{classes:id39}}\begin{description}
\sphinxlineitem{task}{[}dict{]}
\sphinxAtStartPar
Dictionary with all the task folders, x and y positions

\sphinxlineitem{filename}{[}str{]}
\sphinxAtStartPar
Filename of the file in each task folder

\sphinxlineitem{path}{[}str{]}
\sphinxAtStartPar
Path where all the files are

\sphinxlineitem{nx}{[}int{]}
\sphinxAtStartPar
how many results are read in x

\sphinxlineitem{ny}{[}int{]}
\sphinxAtStartPar
how many results are read in y

\end{description}


\paragraph{Return}
\label{\detokenize{classes:id40}}
\sphinxAtStartPar
class with the parameters

\end{fulllineitems}

\index{read\_results\_MC() (profile\_stk.Profile method)@\spxentry{read\_results\_MC()}\spxextra{profile\_stk.Profile method}}

\begin{fulllineitems}
\phantomsection\label{\detokenize{classes:profile_stk.Profile.read_results_MC}}
\pysigstartsignatures
\pysiglinewithargsret{\sphinxbfcode{\sphinxupquote{read\_results\_MC}}}{\sphinxparam{\DUrole{n}{path}}\sphinxparamcomma \sphinxparam{\DUrole{n}{tasks}}\sphinxparamcomma \sphinxparam{\DUrole{n}{filename}}}{}
\pysigstopsignatures
\sphinxAtStartPar
Reads all the profiles of the synthesis or inversion
\begin{description}
\sphinxlineitem{config}{[}dict{]}
\sphinxAtStartPar
Config information

\sphinxlineitem{tasks}{[}dict{]}
\sphinxAtStartPar
Dictionary with the folder names

\sphinxlineitem{Type}{[}string, optional{]}
\sphinxAtStartPar
Indicating if synthesis or inversion. Default: ‘syn’

\end{description}

\end{fulllineitems}

\index{set\_dim() (profile\_stk.Profile method)@\spxentry{set\_dim()}\spxextra{profile\_stk.Profile method}}

\begin{fulllineitems}
\phantomsection\label{\detokenize{classes:profile_stk.Profile.set_dim}}
\pysigstartsignatures
\pysiglinewithargsret{\sphinxbfcode{\sphinxupquote{set\_dim}}}{\sphinxparam{\DUrole{n}{nx}}\sphinxparamcomma \sphinxparam{\DUrole{n}{ny}}\sphinxparamcomma \sphinxparam{\DUrole{n}{nw}}}{}
\pysigstopsignatures
\sphinxAtStartPar
Sets the dimensions if no data is loaded yet


\paragraph{Parameter}
\label{\detokenize{classes:id41}}\begin{description}
\sphinxlineitem{nx}{[}int{]}
\sphinxAtStartPar
Number of Models in x

\sphinxlineitem{ny}{[}int{]}
\sphinxAtStartPar
Number of Models in y

\sphinxlineitem{nw}{[}int{]}
\sphinxAtStartPar
Number of wavelength points

\end{description}

\end{fulllineitems}

\index{write() (profile\_stk.Profile method)@\spxentry{write()}\spxextra{profile\_stk.Profile method}}

\begin{fulllineitems}
\phantomsection\label{\detokenize{classes:profile_stk.Profile.write}}
\pysigstartsignatures
\pysiglinewithargsret{\sphinxbfcode{\sphinxupquote{write}}}{\sphinxparam{\DUrole{n}{fname}}\sphinxparamcomma \sphinxparam{\DUrole{n}{fmt\_type=\textless{}class \textquotesingle{}numpy.float32\textquotesingle{}\textgreater{}}}}{}
\pysigstopsignatures
\sphinxAtStartPar
Write data into a binary fortran file


\paragraph{Parameter}
\label{\detokenize{classes:id42}}\begin{description}
\sphinxlineitem{fname}{[}str{]}
\sphinxAtStartPar
File name

\sphinxlineitem{fmt\_type}{[}type{]}
\sphinxAtStartPar
type which is used to save it =\textgreater{} numpy.float64 used

\end{description}

\end{fulllineitems}

\index{write\_profile() (profile\_stk.Profile method)@\spxentry{write\_profile()}\spxextra{profile\_stk.Profile method}}

\begin{fulllineitems}
\phantomsection\label{\detokenize{classes:profile_stk.Profile.write_profile}}
\pysigstartsignatures
\pysiglinewithargsret{\sphinxbfcode{\sphinxupquote{write\_profile}}}{\sphinxparam{\DUrole{n}{filename}}\sphinxparamcomma \sphinxparam{\DUrole{n}{x}}\sphinxparamcomma \sphinxparam{\DUrole{n}{y}}\sphinxparamcomma \sphinxparam{\DUrole{n}{Grid}}}{}
\pysigstopsignatures
\sphinxAtStartPar
Writes data to profiles as described in the config file
Note to call cut\_to\_wave, otherwise it writes the wrong profiles!


\paragraph{Parameter}
\label{\detokenize{classes:id43}}\begin{description}
\sphinxlineitem{filename}{[}string{]}
\sphinxAtStartPar
String containing the output path of the profile

\sphinxlineitem{x}{[}int{]}
\sphinxAtStartPar
Position in x

\sphinxlineitem{y}{[}int{]}
\sphinxAtStartPar
Position in y

\sphinxlineitem{Grid}{[}string{]}
\sphinxAtStartPar
Grid file

\end{description}

\end{fulllineitems}

\index{write\_profile\_mc() (profile\_stk.Profile method)@\spxentry{write\_profile\_mc()}\spxextra{profile\_stk.Profile method}}

\begin{fulllineitems}
\phantomsection\label{\detokenize{classes:profile_stk.Profile.write_profile_mc}}
\pysigstartsignatures
\pysiglinewithargsret{\sphinxbfcode{\sphinxupquote{write\_profile\_mc}}}{\sphinxparam{\DUrole{n}{filename}}\sphinxparamcomma \sphinxparam{\DUrole{n}{x}}}{}
\pysigstopsignatures
\sphinxAtStartPar
Writes data to profiles as described in the config file


\paragraph{Parameter}
\label{\detokenize{classes:id44}}\begin{description}
\sphinxlineitem{filename}{[}string{]}
\sphinxAtStartPar
String containing the output path of the profile

\sphinxlineitem{x}{[}int{]}
\sphinxAtStartPar
Position in x

\sphinxlineitem{y}{[}int{]}
\sphinxAtStartPar
Position in y

\sphinxlineitem{Grid}{[}string{]}
\sphinxAtStartPar
Grid file

\end{description}

\end{fulllineitems}

\index{cut\_to\_map() (profile\_stk.Profile method)@\spxentry{cut\_to\_map()}\spxextra{profile\_stk.Profile method}}

\begin{fulllineitems}
\phantomsection\label{\detokenize{classes:id45}}
\pysigstartsignatures
\pysiglinewithargsret{\sphinxbfcode{\sphinxupquote{cut\_to\_map}}}{\sphinxparam{\DUrole{n}{Map}}}{}
\pysigstopsignatures
\sphinxAtStartPar
Cut the data to a map {[}xmin, xmax, ymin, ymax{]}


\paragraph{Parameters}
\label{\detokenize{classes:id46}}\begin{description}
\sphinxlineitem{Map}{[}list{]}
\sphinxAtStartPar
List with the ranges in pixel in x and y direction

\end{description}

\end{fulllineitems}

\index{cut\_to\_wave() (profile\_stk.Profile method)@\spxentry{cut\_to\_wave()}\spxextra{profile\_stk.Profile method}}

\begin{fulllineitems}
\phantomsection\label{\detokenize{classes:id47}}
\pysigstartsignatures
\pysiglinewithargsret{\sphinxbfcode{\sphinxupquote{cut\_to\_wave}}}{\sphinxparam{\DUrole{n}{range\_wave}}}{}
\pysigstopsignatures
\sphinxAtStartPar
Cut the data to the range in wavelengths


\paragraph{Parameters}
\label{\detokenize{classes:id48}}\begin{description}
\sphinxlineitem{range\_wave}{[}list{]}
\sphinxAtStartPar
List with the ranges from the config file

\end{description}

\end{fulllineitems}

\index{read\_results() (profile\_stk.Profile method)@\spxentry{read\_results()}\spxextra{profile\_stk.Profile method}}

\begin{fulllineitems}
\phantomsection\label{\detokenize{classes:id49}}
\pysigstartsignatures
\pysiglinewithargsret{\sphinxbfcode{\sphinxupquote{read\_results}}}{\sphinxparam{\DUrole{n}{task}}\sphinxparamcomma \sphinxparam{\DUrole{n}{filename}}\sphinxparamcomma \sphinxparam{\DUrole{n}{path}}\sphinxparamcomma \sphinxparam{\DUrole{n}{nx}}\sphinxparamcomma \sphinxparam{\DUrole{n}{ny}}}{}
\pysigstopsignatures
\sphinxAtStartPar
Reads all the errors from the inversion


\paragraph{Parameter}
\label{\detokenize{classes:id50}}\begin{description}
\sphinxlineitem{task}{[}dict{]}
\sphinxAtStartPar
Dictionary with all the task folders, x and y positions

\sphinxlineitem{filename}{[}str{]}
\sphinxAtStartPar
Filename of the file in each task folder

\sphinxlineitem{path}{[}str{]}
\sphinxAtStartPar
Path where all the files are

\sphinxlineitem{nx}{[}int{]}
\sphinxAtStartPar
how many results are read in x

\sphinxlineitem{ny}{[}int{]}
\sphinxAtStartPar
how many results are read in y

\end{description}


\paragraph{Return}
\label{\detokenize{classes:id51}}
\sphinxAtStartPar
class with the parameters

\end{fulllineitems}

\index{read\_results\_MC() (profile\_stk.Profile method)@\spxentry{read\_results\_MC()}\spxextra{profile\_stk.Profile method}}

\begin{fulllineitems}
\phantomsection\label{\detokenize{classes:id52}}
\pysigstartsignatures
\pysiglinewithargsret{\sphinxbfcode{\sphinxupquote{read\_results\_MC}}}{\sphinxparam{\DUrole{n}{path}}\sphinxparamcomma \sphinxparam{\DUrole{n}{tasks}}\sphinxparamcomma \sphinxparam{\DUrole{n}{filename}}}{}
\pysigstopsignatures
\sphinxAtStartPar
Reads all the profiles of the synthesis or inversion
\begin{description}
\sphinxlineitem{config}{[}dict{]}
\sphinxAtStartPar
Config information

\sphinxlineitem{tasks}{[}dict{]}
\sphinxAtStartPar
Dictionary with the folder names

\sphinxlineitem{Type}{[}string, optional{]}
\sphinxAtStartPar
Indicating if synthesis or inversion. Default: ‘syn’

\end{description}

\end{fulllineitems}

\index{set\_dim() (profile\_stk.Profile method)@\spxentry{set\_dim()}\spxextra{profile\_stk.Profile method}}

\begin{fulllineitems}
\phantomsection\label{\detokenize{classes:id53}}
\pysigstartsignatures
\pysiglinewithargsret{\sphinxbfcode{\sphinxupquote{set\_dim}}}{\sphinxparam{\DUrole{n}{nx}}\sphinxparamcomma \sphinxparam{\DUrole{n}{ny}}\sphinxparamcomma \sphinxparam{\DUrole{n}{nw}}}{}
\pysigstopsignatures
\sphinxAtStartPar
Sets the dimensions if no data is loaded yet


\paragraph{Parameter}
\label{\detokenize{classes:id54}}\begin{description}
\sphinxlineitem{nx}{[}int{]}
\sphinxAtStartPar
Number of Models in x

\sphinxlineitem{ny}{[}int{]}
\sphinxAtStartPar
Number of Models in y

\sphinxlineitem{nw}{[}int{]}
\sphinxAtStartPar
Number of wavelength points

\end{description}

\end{fulllineitems}

\index{write() (profile\_stk.Profile method)@\spxentry{write()}\spxextra{profile\_stk.Profile method}}

\begin{fulllineitems}
\phantomsection\label{\detokenize{classes:id55}}
\pysigstartsignatures
\pysiglinewithargsret{\sphinxbfcode{\sphinxupquote{write}}}{\sphinxparam{\DUrole{n}{fname}}\sphinxparamcomma \sphinxparam{\DUrole{n}{fmt\_type=\textless{}class \textquotesingle{}numpy.float32\textquotesingle{}\textgreater{}}}}{}
\pysigstopsignatures
\sphinxAtStartPar
Write data into a binary fortran file


\paragraph{Parameter}
\label{\detokenize{classes:id56}}\begin{description}
\sphinxlineitem{fname}{[}str{]}
\sphinxAtStartPar
File name

\sphinxlineitem{fmt\_type}{[}type{]}
\sphinxAtStartPar
type which is used to save it =\textgreater{} numpy.float64 used

\end{description}

\end{fulllineitems}

\index{write\_profile() (profile\_stk.Profile method)@\spxentry{write\_profile()}\spxextra{profile\_stk.Profile method}}

\begin{fulllineitems}
\phantomsection\label{\detokenize{classes:id57}}
\pysigstartsignatures
\pysiglinewithargsret{\sphinxbfcode{\sphinxupquote{write\_profile}}}{\sphinxparam{\DUrole{n}{filename}}\sphinxparamcomma \sphinxparam{\DUrole{n}{x}}\sphinxparamcomma \sphinxparam{\DUrole{n}{y}}\sphinxparamcomma \sphinxparam{\DUrole{n}{Grid}}}{}
\pysigstopsignatures
\sphinxAtStartPar
Writes data to profiles as described in the config file
Note to call cut\_to\_wave, otherwise it writes the wrong profiles!


\paragraph{Parameter}
\label{\detokenize{classes:id58}}\begin{description}
\sphinxlineitem{filename}{[}string{]}
\sphinxAtStartPar
String containing the output path of the profile

\sphinxlineitem{x}{[}int{]}
\sphinxAtStartPar
Position in x

\sphinxlineitem{y}{[}int{]}
\sphinxAtStartPar
Position in y

\sphinxlineitem{Grid}{[}string{]}
\sphinxAtStartPar
Grid file

\end{description}

\end{fulllineitems}

\index{write\_profile\_mc() (profile\_stk.Profile method)@\spxentry{write\_profile\_mc()}\spxextra{profile\_stk.Profile method}}

\begin{fulllineitems}
\phantomsection\label{\detokenize{classes:id59}}
\pysigstartsignatures
\pysiglinewithargsret{\sphinxbfcode{\sphinxupquote{write\_profile\_mc}}}{\sphinxparam{\DUrole{n}{filename}}\sphinxparamcomma \sphinxparam{\DUrole{n}{x}}}{}
\pysigstopsignatures
\sphinxAtStartPar
Writes data to profiles as described in the config file


\paragraph{Parameter}
\label{\detokenize{classes:id60}}\begin{description}
\sphinxlineitem{filename}{[}string{]}
\sphinxAtStartPar
String containing the output path of the profile

\sphinxlineitem{x}{[}int{]}
\sphinxAtStartPar
Position in x

\sphinxlineitem{y}{[}int{]}
\sphinxAtStartPar
Position in y

\sphinxlineitem{Grid}{[}string{]}
\sphinxAtStartPar
Grid file

\end{description}

\end{fulllineitems}


\end{fulllineitems}


\sphinxstepscope


\subsection{API}
\label{\detokenize{api:api}}\label{\detokenize{api::doc}}

\begin{savenotes}\sphinxattablestart
\sphinxthistablewithglobalstyle
\sphinxthistablewithnovlinesstyle
\centering
\begin{tabulary}{\linewidth}[t]{\X{1}{2}\X{1}{2}}
\sphinxtoprule
\sphinxtableatstartofbodyhook
\sphinxAtStartPar
{\hyperref[\detokenize{generated/create_config:module-create_config}]{\sphinxcrossref{\sphinxcode{\sphinxupquote{create\_config}}}}}
&
\sphinxAtStartPar
Create a config file as expected for the inversion
\\
\sphinxhline
\sphinxAtStartPar
{\hyperref[\detokenize{generated/sir:module-sir}]{\sphinxcrossref{\sphinxcode{\sphinxupquote{sir}}}}}
&
\sphinxAtStartPar
Library for repeating functions for analyzing and/or plotting SIR data
\\
\sphinxhline
\sphinxAtStartPar
{\hyperref[\detokenize{generated/model:module-model}]{\sphinxcrossref{\sphinxcode{\sphinxupquote{model}}}}}
&
\sphinxAtStartPar
Class Model with all the tools to read and write the models
\\
\sphinxhline
\sphinxAtStartPar
{\hyperref[\detokenize{generated/profile_stk:module-profile_stk}]{\sphinxcrossref{\sphinxcode{\sphinxupquote{profile\_stk}}}}}
&
\sphinxAtStartPar
Class Profile with all the tools to read and write the Stokes Profiles
\\
\sphinxbottomrule
\end{tabulary}
\sphinxtableafterendhook\par
\sphinxattableend\end{savenotes}

\sphinxstepscope


\subsubsection{create\_config}
\label{\detokenize{generated/create_config:module-create_config}}\label{\detokenize{generated/create_config:create-config}}\label{\detokenize{generated/create_config::doc}}\index{module@\spxentry{module}!create\_config@\spxentry{create\_config}}\index{create\_config@\spxentry{create\_config}!module@\spxentry{module}}
\sphinxAtStartPar
Create a config file as expected for the inversion
\subsubsection*{Functions}


\begin{savenotes}\sphinxattablestart
\sphinxthistablewithglobalstyle
\sphinxthistablewithnovlinesstyle
\centering
\begin{tabulary}{\linewidth}[t]{\X{1}{2}\X{1}{2}}
\sphinxtoprule
\sphinxtableatstartofbodyhook
\sphinxAtStartPar
{\hyperref[\detokenize{functions:create_config.config_1C}]{\sphinxcrossref{\sphinxcode{\sphinxupquote{config\_1C}}}}}()
&
\sphinxAtStartPar
Creates a config file for the 1 Component Inversion by asking questions.
\\
\sphinxhline
\sphinxAtStartPar
{\hyperref[\detokenize{functions:create_config.config_2C}]{\sphinxcrossref{\sphinxcode{\sphinxupquote{config\_2C}}}}}()
&
\sphinxAtStartPar
Creates a config file for the 2 Components Inversion by asking questions.
\\
\sphinxhline
\sphinxAtStartPar
{\hyperref[\detokenize{functions:create_config.config_MC}]{\sphinxcrossref{\sphinxcode{\sphinxupquote{config\_MC}}}}}()
&
\sphinxAtStartPar
Creates a config file for the Monte\sphinxhyphen{}Carlo simulation by asking questions.
\\
\sphinxhline
\sphinxAtStartPar
\sphinxcode{\sphinxupquote{help}}()
&
\sphinxAtStartPar

\\
\sphinxbottomrule
\end{tabulary}
\sphinxtableafterendhook\par
\sphinxattableend\end{savenotes}

\sphinxstepscope


\subsubsection{sir}
\label{\detokenize{generated/sir:module-sir}}\label{\detokenize{generated/sir:sir}}\label{\detokenize{generated/sir::doc}}\index{module@\spxentry{module}!sir@\spxentry{sir}}\index{sir@\spxentry{sir}!module@\spxentry{module}}
\sphinxAtStartPar
Library for repeating functions for analyzing and/or plotting SIR data
\subsubsection*{Functions}


\begin{savenotes}\sphinxattablestart
\sphinxthistablewithglobalstyle
\sphinxthistablewithnovlinesstyle
\centering
\begin{tabulary}{\linewidth}[t]{\X{1}{2}\X{1}{2}}
\sphinxtoprule
\sphinxtableatstartofbodyhook
\sphinxAtStartPar
{\hyperref[\detokenize{functions:sir.list_to_string}]{\sphinxcrossref{\sphinxcode{\sphinxupquote{list\_to\_string}}}}}(temp{[}, let{]})
&
\sphinxAtStartPar
Convert a list to a string
\\
\sphinxhline
\sphinxAtStartPar
{\hyperref[\detokenize{functions:sir.option}]{\sphinxcrossref{\sphinxcode{\sphinxupquote{option}}}}}(text1, text2)
&
\sphinxAtStartPar
Print an option in a help page
\\
\sphinxhline
\sphinxAtStartPar
{\hyperref[\detokenize{functions:sir.read_chi2}]{\sphinxcrossref{\sphinxcode{\sphinxupquote{read\_chi2}}}}}(filename{[}, task{]})
&
\sphinxAtStartPar
Reads the last chi value in a inv.chi file
\\
\sphinxhline
\sphinxAtStartPar
{\hyperref[\detokenize{functions:sir.read_chi2s}]{\sphinxcrossref{\sphinxcode{\sphinxupquote{read\_chi2s}}}}}(conf, tasks)
&
\sphinxAtStartPar
Reads all the chi2 from the inversion
\\
\sphinxhline
\sphinxAtStartPar
{\hyperref[\detokenize{functions:sir.read_config}]{\sphinxcrossref{\sphinxcode{\sphinxupquote{read\_config}}}}}(filename{[}, check, change\_config{]})
&
\sphinxAtStartPar
Reads a config file for the inversion
\\
\sphinxhline
\sphinxAtStartPar
{\hyperref[\detokenize{functions:sir.read_control}]{\sphinxcrossref{\sphinxcode{\sphinxupquote{read\_control}}}}}(filename)
&
\sphinxAtStartPar
Reads a control file in the scheme SIR expects it.
\\
\sphinxhline
\sphinxAtStartPar
{\hyperref[\detokenize{functions:sir.read_grid}]{\sphinxcrossref{\sphinxcode{\sphinxupquote{read\_grid}}}}}(filename)
&
\sphinxAtStartPar
Reads the grid file
\\
\sphinxhline
\sphinxAtStartPar
{\hyperref[\detokenize{functions:sir.read_line}]{\sphinxcrossref{\sphinxcode{\sphinxupquote{read\_line}}}}}(filename)
&
\sphinxAtStartPar
Reads the line file
\\
\sphinxhline
\sphinxAtStartPar
{\hyperref[\detokenize{functions:sir.read_model}]{\sphinxcrossref{\sphinxcode{\sphinxupquote{read\_model}}}}}(filename)
&
\sphinxAtStartPar
Reads a model file and returns all parameters
\\
\sphinxhline
\sphinxAtStartPar
{\hyperref[\detokenize{functions:sir.read_profile}]{\sphinxcrossref{\sphinxcode{\sphinxupquote{read\_profile}}}}}(filename{[}, num{]})
&
\sphinxAtStartPar
Reads the first LINE data from a profile computed by SIR
\\
\sphinxhline
\sphinxAtStartPar
{\hyperref[\detokenize{functions:sir.write_config}]{\sphinxcrossref{\sphinxcode{\sphinxupquote{write\_config}}}}}(File, conf)
&
\sphinxAtStartPar
Writes a config file with the information provided as a dictionary for the mode 1C
\\
\sphinxhline
\sphinxAtStartPar
{\hyperref[\detokenize{functions:sir.write_config_1c}]{\sphinxcrossref{\sphinxcode{\sphinxupquote{write\_config\_1c}}}}}(File, conf)
&
\sphinxAtStartPar
Writes a config file with the information provided as a dictionary for the mode 1C
\\
\sphinxhline
\sphinxAtStartPar
{\hyperref[\detokenize{functions:sir.write_config_2c}]{\sphinxcrossref{\sphinxcode{\sphinxupquote{write\_config\_2c}}}}}(File, conf)
&
\sphinxAtStartPar
Writes a config file with the information provided as a dictionary
\\
\sphinxhline
\sphinxAtStartPar
{\hyperref[\detokenize{functions:sir.write_config_mc}]{\sphinxcrossref{\sphinxcode{\sphinxupquote{write\_config\_mc}}}}}(File, conf)
&
\sphinxAtStartPar
Writes a config file with the information provided as a dictionary
\\
\sphinxhline
\sphinxAtStartPar
{\hyperref[\detokenize{functions:sir.write_control_1c}]{\sphinxcrossref{\sphinxcode{\sphinxupquote{write\_control\_1c}}}}}(filename, conf)
&
\sphinxAtStartPar
Writes a control file in the scheme SIR expects it.
\\
\sphinxhline
\sphinxAtStartPar
{\hyperref[\detokenize{functions:sir.write_control_2c}]{\sphinxcrossref{\sphinxcode{\sphinxupquote{write\_control\_2c}}}}}(filename, conf)
&
\sphinxAtStartPar
Writes a control file in the scheme SIR expects it.
\\
\sphinxhline
\sphinxAtStartPar
{\hyperref[\detokenize{functions:sir.write_control_mc}]{\sphinxcrossref{\sphinxcode{\sphinxupquote{write\_control\_mc}}}}}(filename, conf{[}, Type{]})
&
\sphinxAtStartPar
Writes a control file in the scheme SIR expects it.
\\
\sphinxhline
\sphinxAtStartPar
{\hyperref[\detokenize{functions:sir.write_grid}]{\sphinxcrossref{\sphinxcode{\sphinxupquote{write\_grid}}}}}(conf, waves{[}, filename{]})
&
\sphinxAtStartPar
Writes the Grid file with data from the config file
\\
\sphinxhline
\sphinxAtStartPar
{\hyperref[\detokenize{functions:sir.write_grid_mc}]{\sphinxcrossref{\sphinxcode{\sphinxupquote{write\_grid\_mc}}}}}(conf{[}, filename{]})
&
\sphinxAtStartPar
Writes the Grid file with data from the config file
\\
\sphinxhline
\sphinxAtStartPar
{\hyperref[\detokenize{functions:sir.write_model}]{\sphinxcrossref{\sphinxcode{\sphinxupquote{write\_model}}}}}(filename, Header, log\_tau, T, ...)
&
\sphinxAtStartPar
Write a model with the given data in a specific format.
\\
\sphinxhline
\sphinxAtStartPar
{\hyperref[\detokenize{functions:sir.write_profile}]{\sphinxcrossref{\sphinxcode{\sphinxupquote{write\_profile}}}}}(filename, profiles, pos)
&
\sphinxAtStartPar
Write a profile for a specific model number to a file
\\
\sphinxbottomrule
\end{tabulary}
\sphinxtableafterendhook\par
\sphinxattableend\end{savenotes}

\sphinxstepscope


\subsubsection{model}
\label{\detokenize{generated/model:module-model}}\label{\detokenize{generated/model:model}}\label{\detokenize{generated/model::doc}}\index{module@\spxentry{module}!model@\spxentry{model}}\index{model@\spxentry{model}!module@\spxentry{module}}
\sphinxAtStartPar
Class Model with all the tools to read and write the models
\subsubsection*{Functions}


\begin{savenotes}\sphinxattablestart
\sphinxthistablewithglobalstyle
\sphinxthistablewithnovlinesstyle
\centering
\begin{tabulary}{\linewidth}[t]{\X{1}{2}\X{1}{2}}
\sphinxtoprule
\sphinxtableatstartofbodyhook
\sphinxAtStartPar
{\hyperref[\detokenize{functions:model.read_model}]{\sphinxcrossref{\sphinxcode{\sphinxupquote{read\_model}}}}}(filename)
&
\sphinxAtStartPar

\\
\sphinxbottomrule
\end{tabulary}
\sphinxtableafterendhook\par
\sphinxattableend\end{savenotes}
\subsubsection*{Classes}


\begin{savenotes}\sphinxattablestart
\sphinxthistablewithglobalstyle
\sphinxthistablewithnovlinesstyle
\centering
\begin{tabulary}{\linewidth}[t]{\X{1}{2}\X{1}{2}}
\sphinxtoprule
\sphinxtableatstartofbodyhook
\sphinxAtStartPar
{\hyperref[\detokenize{classes:model.Model}]{\sphinxcrossref{\sphinxcode{\sphinxupquote{Model}}}}}({[}nx, ny, nval{]})
&
\sphinxAtStartPar
Class containing the models of a simulation
\\
\sphinxbottomrule
\end{tabulary}
\sphinxtableafterendhook\par
\sphinxattableend\end{savenotes}

\sphinxstepscope


\subsubsection{profile\_stk}
\label{\detokenize{generated/profile_stk:module-profile_stk}}\label{\detokenize{generated/profile_stk:profile-stk}}\label{\detokenize{generated/profile_stk::doc}}\index{module@\spxentry{module}!profile\_stk@\spxentry{profile\_stk}}\index{profile\_stk@\spxentry{profile\_stk}!module@\spxentry{module}}
\sphinxAtStartPar
Class Profile with all the tools to read and write the Stokes Profiles
\subsubsection*{Functions}


\begin{savenotes}\sphinxattablestart
\sphinxthistablewithglobalstyle
\sphinxthistablewithnovlinesstyle
\centering
\begin{tabulary}{\linewidth}[t]{\X{1}{2}\X{1}{2}}
\sphinxtoprule
\sphinxtableatstartofbodyhook
\sphinxAtStartPar
\sphinxcode{\sphinxupquote{read\_profile}}(file)
&
\sphinxAtStartPar
Reads a profile and returns a class
\\
\sphinxbottomrule
\end{tabulary}
\sphinxtableafterendhook\par
\sphinxattableend\end{savenotes}
\subsubsection*{Classes}


\begin{savenotes}\sphinxattablestart
\sphinxthistablewithglobalstyle
\sphinxthistablewithnovlinesstyle
\centering
\begin{tabulary}{\linewidth}[t]{\X{1}{2}\X{1}{2}}
\sphinxtoprule
\sphinxtableatstartofbodyhook
\sphinxAtStartPar
{\hyperref[\detokenize{classes:profile_stk.Profile}]{\sphinxcrossref{\sphinxcode{\sphinxupquote{Profile}}}}}({[}nx, ny, nw{]})
&
\sphinxAtStartPar
Class containing the models of a simulation
\\
\sphinxbottomrule
\end{tabulary}
\sphinxtableafterendhook\par
\sphinxattableend\end{savenotes}


\renewcommand{\indexname}{Python Module Index}
\begin{sphinxtheindex}
\let\bigletter\sphinxstyleindexlettergroup
\bigletter{c}
\item\relax\sphinxstyleindexentry{create\_config}\sphinxstyleindexpageref{generated/create_config:\detokenize{module-create_config}}
\indexspace
\bigletter{m}
\item\relax\sphinxstyleindexentry{model}\sphinxstyleindexpageref{generated/model:\detokenize{module-model}}
\indexspace
\bigletter{p}
\item\relax\sphinxstyleindexentry{profile\_stk}\sphinxstyleindexpageref{generated/profile_stk:\detokenize{module-profile_stk}}
\indexspace
\bigletter{s}
\item\relax\sphinxstyleindexentry{sir}\sphinxstyleindexpageref{generated/sir:\detokenize{module-sir}}
\end{sphinxtheindex}

\renewcommand{\indexname}{Index}
\printindex
\end{document}